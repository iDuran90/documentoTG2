\thispagestyle{fancy}
\pagenumbering{roman}
\setcounter{page}{3}

\begin{huge}
\textbf{Nota de aceptación}
\end{huge} 
\vfill
\begin{Large}
\begin{flushright}
\rule{100mm}{0.05mm}\\
\rule{100mm}{0.05mm}\\
\rule{100mm}{0.05mm}\\
\rule{100mm}{0.05mm}\\
\rule{100mm}{0.05mm}\\
\textbf{Presidente
del Jurado}\\[2.0cm]
\rule{100mm}{0.05mm}\\\textbf{Jurado 1}\vfill
\rule{100mm}{0.05mm}\\\textbf{Jurado 2}\vfill
\end{flushright}
\begin{flushleft}
\large \textbf{Cali, Enero de 2014}
\end{flushleft}
\end{Large}
\newpage

\selectlanguage{spanish}

\begin{abstract}
\thispagestyle{fancy}
\setcounter{page}{4}
\justifying
El proceso de minería de datos, que permite descubrir nuevo y útil conocimiento a partir del análisis exhaustivo de información recolectada en un cierto intervalo del tiempo, puede ser aplicado a distintos contextos de la vida cotidiana. Para el caso de estudio que se desarrolló durante este proyecto, el contexto fue la educación. Específicamente se planteó lograr predecir puntajes de la prueba Saber 11\degree \ desarrollada por el Instituto Colombiano para la Evaluación de la Educación (ICFES).

En este trabajo se presenta el desarrollo del proyecto de minería de datos, se usó como fuentes de información las bases de datos del ICFES, las cual contienen información histórica sobre resultados e información personal de los evaluados.

Como resultado final se presenta la aplicación web PrediXaber11, donde las personas pueden acceder y ejecutar consultas. La aplicación usará la información suministrada del evaluado para entregar los posibles resultados en cada una de las áreas académicas evaluadas por el ICFES.\\

Palabras clave: Minería de datos, Saber 11\degree, Aplicación Web, KDD, Weka, ICFES, Java.
\end{abstract}
\selectlanguage{spanish}