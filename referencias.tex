\begin{thebibliography}{Referencias}
\bibitem{key-1}\foreignlanguage{english}{Periódico
El Colombiano, \textquotedblleft{}Leve mejoría en pruebas Saber 11\textquotedblright{}.
{[}artículo en Internet{]}}. \url{http://www.elcolombiano.com/BancoConocimiento/L/leve_mejoria_en_pruebas_saber_11/leve_mejoria_en_pruebas_saber_11.asp}\foreignlanguage{english}{
{[}Consulta: 24 agosto de 2012{]}.}

\bibitem{key-2}Periódico El Tiempo, \textquotedblleft{}El 45\% de
los colegios presentó bajo rendimiento en pruebas Saber 11\textquotedblright{}.
{[}artículo en Internet{]}\foreignlanguage{spanish}{. }\url{http://www.eltiempo.com/vida-de-hoy/educacion/ARTICULO-WEB-NEW\_NOTA\_INTERIOR-8384822.html}\foreignlanguage{spanish}{
}{[}Consulta: 24 agosto de 2012{]}.

\bibitem{key-3}Revista Dinero, \textquotedblleft{}Las pruebas del
Icfes no son el único indicador\textquotedblright{}. {[}artículo en
Internet{]}   \url{http://m.dinero.com/edicion-impresa/caratula/articulo/las-pruebas-del-icfes-no-unico-indicador/139069}
{[}Consulta: 14 marzo de 2012{]}.

\bibitem{key-4}Departamento Del Valle Del Cauca, Secretaria De Educación.
Informe Ejecutivo Análisis Pruebas Saber 5º, 9º Y 11º. Santiago de
Cali, Febrero 15 de 2012.

\bibitem{key-50}\foreignlanguage{spanish}{Hernández Orallo José,
Ramírez Quintana Ma. José, Ferri, Ramírez César, Introducción a la
minería de datos, Person Educación, S.A. Madrid 2004, ISBN: 978-84-205-4091-7.}

\selectlanguage{spanish}%
\bibitem{key-60}José C. Riquelme, Roberto Ruiz, Karina Gilbert, \textquotedblleft{}Minería
de Datos: Conceptos y Tendencias,\textquotedblright{} \emph{Revista
Iberoamericana de Inteligencia Artificial}, No. 29 (2006), pp. 11-18.

\bibitem{key-70}Téllez A., Extracción de Información con Algoritmos
de Clasificación {[}Tesis de Maestría{]}. Tonantzintla, Puebla, México.
Instituto Nacional de Astrofísica, Óptica y Electrónica. 2005.

\bibitem{key-80}ICFES, Presentación. {[}artículo en Internet{]} \url{http://www2.icfes.gov.co/informacion-institucional/informacion-general}
{[}Consulta: 4 junio de 2012{]}.

\bibitem{key-90}C.S.R. Prabhu, Data Warehousing Concepts, Techniques
Products and Applications, Prentice-Hall, India 2006, ISBN: 81-203-2068-9.

\bibitem{key-100}Timarán Pereira, Ricardo. Una lectura sobre deserción
universitaria en estudiantes de pregrado desde la perspectiva de la
minería de datos.Revista Científica Guillermo de Ockham, vol. 8, núm.
1, enero-junio, 2010, pp. 121-130.

\bibitem{key-110}Álvaro Jiménez Galindo, Hugo Álvarez García. Minería
de Datos en la Educación. Universidad Carlos III de Madrid.

\bibitem{key-120}Dr Ray Hoare, Using CHAID for classication problems,
en New Zealand Statistical Association 2004 conference, Wellington.

\bibitem{key-130}Sergio Valero Orea, Alejandro Salvador Vargas, Marcela
García Alonso. Minería de datos: predicción de la deserción escolar
mediante el algoritmo de árboles de decisión y el algoritmo de los
k vecinos más cercanos. Universidad Tecnológica de Izúcar de Matamoros,
Izúcar de Matamoros, Puebla, México. Recursos digitales para la educación
y la cultura volumen Kaambal. ISBN Volumen: 978-607-95446-1-4.

\bibitem{key-140}Erika Rodallegas Ramos, Areli Torres González, Beatriz
B. Gaona Couto, Erick Gastelloú Hernández, Rafael A. Lez ama Morales,
Sergio Valero Orea. Modelo predictivo para la determinación de causas
de reprobación mediante Minería de Datos. Universidad Tecnológica
de Izúcar de Matamoros, Mexico. Recursos digitales para la educación
y la cultura volumen Kaambal. ISBN Volumen: 978-607-95446-1-4.

\bibitem{key-150}Elena Gervilla García, Rafael Jiménez López, Juan
José Montaño Moreno, Albert Sesé Abad, Berta Cajal Blasco, Alfonso
Palmer Pol. La metodología del Data Mining. Una aplicación al consumo
de alcohol en adolescentes. Área de Metodología de las Ciencias del
Comportamiento. Departamento de Psicología. Universitat de les Illes
Balears. Adicciones, 2009, Vol. 21 Núm. 1, Págs. 65-80.

\bibitem{key-160}T.Jyothirmayi et. al., An Algorithm for Better Decision
Tree, International Journal on Computer Science and Engineering Vol.
02, No. 09, 2010, 2827-2830.

\bibitem{key-170}\foreignlanguage{spanish}{Ross Quinlan, C4.5: Programs for Machine Learning, Morgan Kaufmann Publishers. San Mateo, CA 1993, ISBN: 1-55860-238-0.}

\bibitem{key-180}D. Aha, D. Kibler, Instance-based learning algorithms, Machine Learning Vol.
06, 1991, 37-66.

\bibitem{key-190}George H. John, Pat Langley, Estimating Continuous Distributions in Bayesian Classifiers, Eleventh Conference on Uncertainty in Artificial Intelligence, Morgan Kaufmann Publishers. San Mateo, CA 1995, 338-345, ISBN: 1-55860-385-9.

\bibitem{key-200}J. Park, I. W. Sandberg, Universal Approximation Using Radial-Basis-Function
Networks, Neural Computation Vol.
03, 1991, 246-257.

\bibitem{key-210}\foreignlanguage{spanish}{Ian H. Witten, Eibe Frank, Mark A. Hall, Data Mining
Practical Machine Learning
Tools and Techniques - 3rd ed., Morgan Kaufmann Publishers, 2011, ISBN: 978-0-12-374856-0.}

\bibitem{key-220}Jon Louis Bentley, Multidimensional Binary Search Trees Used for Associative Searching, Communications of the ACM Vol. 18, No. 09, 1975, 509-517.

\bibitem{key-230}\foreignlanguage{spanish}{Mitchell T.M., Machine Learning, McGraw-Hill, 1997, ISBN: 0070428077.} \bibitem{key-240}\foreignlanguage{spanish}{Mohamad H. Hassoun, Fundamentals of Artificial Neural Networks, Massachusetts Institute of Technology, 1995, ISBN: 0262514672}
\bibitem{key-250}Guoqiang Peter Zhang, Neural Networks for Classification: A Survey, IEEE Transactions on systems, man, and cybernetics, Part c: Applications and reviews, vol. 30, no. 4, november 2000, 451-462.
\bibitem{key-260}Benjamin Aumaille, J2EE Desarrollo de aplicaciones Web, Ediciones ENI, 2002, ISBN: 2-7460-1912-4.\bibitem{key-270}Nicholas C. Zakas, Jeremy McPeak, Joe Fawcett, Professional Ajax, 2nd Edition, Wiley Publishing, 2007, ISBN: 978-0-470-10949-6.\end{thebibliography}